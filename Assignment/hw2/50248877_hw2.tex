%%% Template originaly created by Karol Kozioł (mail@karol-koziol.net) and modified for ShareLaTeX use

\documentclass[a4paper,11pt]{article}

\usepackage[T1]{fontenc}
\usepackage[utf8]{inputenc}
\usepackage{graphicx}
\usepackage{xcolor}
\usepackage[]{authblk}
\usepackage{subcaption}
\usepackage{multirow}

\renewcommand\familydefault{\sfdefault}
\usepackage{tgheros}
\usepackage[defaultmono]{droidmono}

\usepackage{amsmath,amssymb,amsthm,textcomp}
\usepackage{enumerate}
\usepackage{multicol}
\usepackage{tikz}

\usepackage{geometry}
%\geometry{total={210mm,297mm},
%left=25mm,right=25mm,%
%bindingoffset=0mm, top=20mm,bottom=20mm}


\linespread{1.2}

\newcommand{\linia}{\rule{\linewidth}{0.5pt}}

% custom theorems if needed
\newtheoremstyle{mytheor}
    {1ex}{1ex}{\normalfont}{0pt}{\scshape}{.}{1ex}
    {{\thmname{#1 }}{\thmnumber{#2}}{\thmnote{ (#3)}}}

%\theoremstyle{mytheor}
%\newtheorem{defi}{Definition}

% my own titles
\makeatletter
\renewcommand{\maketitle}{
\begin{center}
\vspace{1ex}
{\LARGE \textsc{\@title}}
\vspace{0.5ex}
\linia\\
\@author \hfill \@date
%\@personNumber \hfill \@email
\vspace{3ex}
\end{center}
}
\makeatother
%%%

% custom footers and headers
\usepackage{fancyhdr}
\pagestyle{fancy}
\lhead{}
\chead{}
\rhead{}
\lfoot{HW 2}
\cfoot{}
\rfoot{Page \thepage}
\renewcommand{\headrulewidth}{0pt}
\renewcommand{\footrulewidth}{0pt}
%

% code listing settings
\usepackage{listings}
\lstset{
    language=Python,
    basicstyle=\ttfamily\small,
    aboveskip={1.0\baselineskip},
    belowskip={1.0\baselineskip},
    columns=fixed,
    extendedchars=true,
    breaklines=true,
    tabsize=4,
    prebreak=\raisebox{0ex}[0ex][0ex]{\ensuremath{\hookleftarrow}},
    frame=lines,
    showtabs=false,
    showspaces=false,
    showstringspaces=false,
    keywordstyle=\color[rgb]{0.627,0.126,0.941},
    commentstyle=\color[rgb]{0.133,0.545,0.133},
    stringstyle=\color[rgb]{01,0,0},
    numbers=left,
    numberstyle=\small,
    stepnumber=1,
    numbersep=10pt,
    captionpos=t,
    escapeinside={\%*}{*)}
}

%%%----------%%%----------%%%----------%%%----------%%%

\begin{document}

\title{Hw 2: Scale-space blob detection}

\author{Avinash Kommineni, 50248877} 
%\personNumber{50248877}
%\email{akommineni@buffalo.edu}
%\personNumber {50248877}
%\email{akommineni@buffalo.edu}
\date{\today}

\maketitle

\section*{Output}

The initialisations are as follows...\\
Nuber of scales = 15\\
Sigma = 1.4\\
Scaling factor = 1.2\\
Threshold = 0.015\\
And the output images are as follows...\\
\includegraphics[width=\textwidth]{hw2/code/1}\\
\includegraphics[width=\textwidth]{hw2/code/2}\\
\includegraphics[width=\textwidth]{hw2/code/3}\\
\includegraphics[scale=0.3]{hw2/code/4}\\
\includegraphics[width=\textwidth]{hw2/code/5}\\
\includegraphics[width=\textwidth]{hw2/code/6}\\
\includegraphics[width=\textwidth]{hw2/code/7}\\
\includegraphics[width=\textwidth]{hw2/code/8}\\
\vfill

\section*{Effecient vs Inefficient}

\begin{center}
	\begin{tabular}{ |c|c|c|c| } 
		\hline
		Time taken (sec) & Image Downscaling & Filter Up-scaling \\
		\hline
		\multirow{8}{7em}{Image run time for the given 4 images and custom 4 images} & 0.063426 & 0.326780 \\ 
		& 0.122941 & 0.502245  \\ 
		& 0.075058 & 0.312024 \\ 
		& 0.055799 & 0.230015 \\ 
		& 1.524554 & 12.276676 \\ 
		& 0.471535 & 3.284120 \\ 
		& 0.180295 & 1.275178 \\
		& 0.327218 & 2.362955 \\
		\hline
	\end{tabular}
\end{center}

\section*{Notes}
\begin{itemize}
	\item The time increases rapidly for the case of filter up-scaling and especially for the for those custom images which are bigger in size.
	\item The time increases for the case of sigma$(\sigma)$=2.0. It increases significantly for the filter up-scaling and marginally for the other one.
	\item 	\begin{tabular}{ |c|c|c|c| } 
		\hline
		Time taken (sec) & Image Downscaling & Filter Up-scaling \\
		\hline
		\multirow{8}{7em}{Image run time for the given 4 images and custom 4 images} & 0.065452 & 0.759418 \\ 
		& 0.096532 & 0.925315  \\ 
		& 0.062356 & 0.580354 \\ 
		& 0.075324 & 0.431651 \\ 
		& 1.474615 & 23.838904 \\ 
		& 0.446366 & 5.750504 \\ 
		& 0.172011 & 2.333773 \\
		& 0.328813 & 3.933201 \\
		\hline
	\end{tabular}
	\vfill
	
	\item The same image with sigma as 1.4 and 2.0\\
	\includegraphics[scale=0.25]{hw2/code/1}\\
	\includegraphics[scale=0.25]{hw2/code/1_1}\\
	Not much differences between two except that the first image has got granular part covered because it's sigma is low.
	\item Thick and sharp edges like the text in $5^{th}$ image is picked up every layer of the filter so the thick resultant circles.
	\item The images produced with filter up-scaling are found in the folder FilterScaled.
	\item While implementing the 3D nonmaximum suppression, the border condition was handled by replicating the first and layer of scale space layers.
	\item Deponding upon the value of sigma, the value of threshold must also be adjusted so that the image is not too over or scarcely populated with circles.
	
\end{itemize}
\end{document}
